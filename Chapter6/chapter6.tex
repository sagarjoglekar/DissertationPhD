\chapter{ Closing notes and outlook for the future }

% **************************** Define Graphics Path **************************

\graphicspath{{Chapter6/plots/}}

\begin{quote}
    ``The unity of all science consists alone in its method, not in its material ... It is not the facts themselves which form science, but the method in which they are dealt with.'' - Karl Pearson
\end{quote}


The guiding principle for this dissertation has been understanding how human subjective opinions can be leveraged for social good. The aim was to advance the understanding of how something as subjective as expression of support or perception of beauty, can be quantified. 
I explored two realms of this problem, one dealing with groups of humans forming communities around a supportive cause, and other looking at a vast set of un-related people expressing their subjective opinions about urban aesthetics. In both cases, the key was the set of methods used to tease out the signatures of these subjective qualities from data. To that end, Karl Pearson's quote is very apt and encapsulates the key contributions of my work.

Working with a framework, of first acquiring and curating \textbf{Data}, then building key abstractions on top to capture the \textbf{Information}, then building metrics to extract the \textbf{Knowledge} allowed me to build a pipeline to work with data from a diverse set of applications. The hope is that this pipeline can be used to generate \textbf{Wisdom} that drives interventions

The work done in my dissertation is going to inspire the vision for my research for the next few years, and I would like to close this journey by walking through a few aspects of what I feel is the path forward.

\section{Open problems}
Throughout this dissertation, I came across interesting problems which I would love to investigate. At the very least, I would like to discuss the problems and illuminate them for further investigation by the community. 

\subsection{Triadic closures in conversation}
Triadic closure has been shown to be an important mechanism in the literature of social networks~\cite{granovetter1977strength,mollenhorst2011shared} through which social ties get established. The mechanism has also been widely explored as a measure to for recommendation systems~\cite{sintos2014using,lou2013learning}.

Despite this wide prevalence in social networks, dialogue structures on reddit in the context of social support seem to exhibit a different behaviour , as seen in Chapter ~\ref{chap:structure_support}. As seen from the results in Appendix ~\ref{chap:app1}, motifs that resemble a triadic closure are very rare in either cases. However the precursors to a triadic closure, such as 201 variants, 111 variants and 021 variants are all expressed with statistical significance in the baseline as well as supportive conversations. Triadic closures have shown to be vital for information diffusion in networks~\cite{babaei2016efficiency}. Hence investigating closures in dialogue structures on reddit could lead to some interesting counter intuitive mechanisms.

\subsection{Colouring ties in dialogue graphs}
In my dissertation I explored the utility of language in capturing the strength of ties in a dialogue graph (as seen in Chapter \ref{chap:structure_support}). However, it is important to note that just measuring alignment between two exchanges might be resulting a loss of crucial information. Linguistically, it would be worthwhile to capture the essence of dialogues along a multi-dimensional scale, since that is how actual exchanges take place. 
To that end, it would be of value to colour the links using affective components of an exchange between two people, such as empathy, affection, trust, anxiety, animosity, friendship etc. These colours could further help us develop methods to detect toxic behaviours online and capture the overall tone of a discussion. 

\subsection{Exploring cultural biases in subjective perception}
An important limitation in using the crowd's opinion to quantify the subjective is the trade-off between data volume and data bias. 
To train any reliable deep learning model, you need a reasonable volume of data to train. This limits the amount of stratification one can do in the crowd opinions along cultural, geographical and social lines. 
Stratifying data further could lead to over-fitted models. But using the bulk of data as one monolithic chuck endangers the model to learn the least common denominator in the subjective preferences. That means the cultural nuances about the concept of beauty that enrich our world are all averaged out. Indeed as seen in Chapter ~\ref{chap:generation} the FaceLift model tends to associate foliage with beauty to a high degree. 
There are methods in the literature to solve these problems, by teasing out biases in the models by partitioning the data in a clever manner.

\section{Future Outlook}
In this Ph.D I aimed at exploring and exploiting the potential of machine learning, to understand how the subjective can be sensed from web scale data. To that end, it is worth discussing about how the methods I developed align with the research I intend to pursue. 

In the future I would like to extend this work along two key dimensions which are based on a shared theme \textit{"Capitalizing on the subjective perceptions to deliver impactful interventions"}.

\subsection{Empathic healthcare}
Improving healthcare to provide an empathic experience to patients in an economical and scalable way is the most crucial challenge of 21st century. 
The world saw an explosive growth in population during the baby boomer generation. This same cohort has now become the largest ageing cohort in the history of the world. On account of this and the rise of chronic diseases, the health care sector has seen unprecedented growth in the past decade\footnote{\url{https://www2.deloitte.com/global/en/pages/life-sciences-and-healthcare/articles/global-health-care-sector-outlook.html}}. With growth, comes the challenges of scale. Despite ever increasing investment\footnote{\url{https://www.bbc.co.uk/news/health-46524257}}, the UK's NHS still is riling under the pressure of rising patient numbers, dwindling staff and longer wait times~\cite{mayor2018nhs}. Longer wait times also imply that doctors are on an average spending less time with the patients. This has taken a serious toll on the doctor patient empathic communication. It has been shown that the doctor patient relationship plays a vital role in accuracy of diagnosis of the disease ,prognosis of the patient and overall satisfaction of the patient~\cite{jagosh2011importance,bensing1991doctor}.

At such a juncture, there is a rising need to solve frictions along these points of contacts for the NHS. At the same time, it is extremely important to provide psyco-social infrastructure for this ageing population, especially at times when the ailments are chronic in nature, and the social support structures are fragile. 
The vision for an empathic healthcare, puts forth a framework where the psycho-social aspects of health are put at the centre of the system, along with the biological well being. This can be done by re-engineering several pipelines, through which a patient engages with the healthcare systems. This means that a treatment plan is not just a relationship between the doctor and the patient, but involves a support structure of both AI-driven agents, peers, online-volunteers and healthcare workers. My work done with the Chronic Obstructive Pulmonary Disease community shows that patients of chronic diseases tend to thrive as a part of an online social support community, and at times can take up the mantle to provide crucial information~\cite{joglekar2018online}. My work with the suicide watch support community also shows that there are quantifiable structures of support which exhibit a patient centric structure of engagement. I foresee that with the help of A.I. models trained on these empathic interactions, we could one day see virtual support groups which are supported by the healthcare provider and allow self and group management of chronic conditions. This would allow the healthcare providers to save costs in terms of lost appointment times for patients of chronic conditions. But at the same time, it would provide a true bio-psycho-social framework for chronic symptom management. Overall, I see that empathic signatures in communications, can be learnt at scale, provided that we can pinpoint where exactly such interactions are happening. And I think my work paves a path forward for that to happen. 


\subsection{Perceptive urbanism}
Urbanism is a term used commonly to describe works that deal with problems and mechanisms that shape an urban environment. There has been a sharp rise in research in the field of urbanism due to the age of open data. Many of these works look at open urban data about socio-economic indicators in conjunction with social outcome variables like health~\cite{scholes2012persistent,venerandi2015measuring}, well being~\cite{cox2017doses}, quality of air~\cite{mitchell2003environmental} etc. 
Some more inter-disciplinary works have also shown that built environment in cities resonate with our personality and perception of qualities like safety, richness and beauty~\cite{de2016safer,dubey2016deep}.Some other studies have also shown that something as simple as presence of green canopy can be correlated with reduced depression related prescriptions~\cite{helbich2018more}.

In the same vein, my work done on crowd based urban design, allowed me to leverage perception of the crowds to improve urban spaces. I believe the most natural extension in this case is utilizing these methods to explore further linkages between our urban environment and our quantified self. Questions like "how does the city influence our mental health?" or "how does built environment influence the air we breath?" are now within the reach of data science. The possibility of linking perception driven metrics in urban spaces, with socio-economic or health indicators is what I define as perceptive urbanism. 
This aspect of urbanism could actually be immensely helpful in developing interventions with minimum costs but maximum impact. 

\section{Conclusion}

In the hindsight, this work has been a result of series of fortunate accidents, which allowed me to explore and exploit interesting topics in the fields of information retrieval, network science and at times social science. Being a computer scientist in today's age, I believe is as much an exercise in a broad understanding of today's social problems, as is it about technical details and methods to solve them. My Ph.D. has given me an unfettered opportunity to spend my time and resources in broadening my horizons. I hope I keep the pursuit alive in the due course of my career. 