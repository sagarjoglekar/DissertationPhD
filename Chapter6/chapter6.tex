\chapter{ Closing notes and outlook for the future }

% **************************** Define Graphics Path **************************

\graphicspath{{Chapter6/plots/}}

\begin{quote}
    ``The unity of all science consists alone in its method, not in its material ... It is not the facts themselves which form science, but the method in which they are dealt with.'' - Karl Pearson
\end{quote}


The guiding principle for this dissertation has been understanding how human subjective opinions can be leveraged for social good. The aim was to advance the understanding of how something as subjective as expression of support or perception of beauty, can be quantified. 
I explored two realms of this problem, one dealing with groups of humans forming communities around a supportive cause, and other looking at a vast set of un-related people expressing their subjective opinions about urban aesthetics. In both cases, the key was the set of methods used to tease out the signatures of these subjective qualities from data. To that end, Karl Pearson's quote is very apt and encapsulates the key contributions of my work.

Working with a framework, of first acquiring and curating \textbf{Data}, then building key abstractions on top to capture the \textbf{Information}, then building metrics to extract the \textbf{Knowledge} allowed me to build a pipeline to work with data from a diverse set of applications. The hope is that this pipeline can be used to generate \textbf{Wisdom} that drives interventions

The work done in my dissertation is going to inspire the vision for my research in the next few years, and I would like to close this journey by walking through a few aspects of what I feel is the path forward.

\section{Empathic Healthcare}

The world saw an explosive growth in population during the baby boomer generation. This same cohort has now become the largest ageing cohort in the history of the world. On account of this and the rise of chronic diseases, the health care sector has seen unprecedented growth in the past decade\footnote{\url{https://www2.deloitte.com/global/en/pages/life-sciences-and-healthcare/articles/global-health-care-sector-outlook.html}}. With growth, comes the challenges of scale. Despite ever increasing investment\footnote{\url{https://www.bbc.co.uk/news/health-46524257}}, the UK's NHS still is riling under the pressure of rising patient numbers, dwindling staff and longer wait times~\cite{mayor2018nhs}. Longer wait times also imply that doctors are on an average spending less time with the patients. This has taken a serious toll on the doctor patient empathic communication. It has been shown that the doctor patient relationship plays a vital role in accuracy of diagnosis of the disease ,prognosis of the patient and overall satisfaction of the patient~\cite{jagosh2011importance,bensing1991doctor}.

At such a juncture, there is a rising need to solve frictions along these points of contacts for NHS. At the same time, it is extremely important to provide psyco-social infrastructure for this ageing population, especially at times when the ailments are chronic in nature, and the social support structures are fragile. 
The vision for an Empathic healthcare, puts forth a framework where the psycho-social aspects of health are put at the centre of the system, along with the biological well being. This can be done by re-engineering several pipelines through with a patient engages with the healthcare systems. This means that a treatment plan is not just a relationship between the doctor and the patient, but involves a support structure of both AI-driven agents, peers, online-volunteers and healthcare workers. My work done with the Chronic Obstructive Pulmonary Disease community, shows that patients of chronic diseases tend to thrive as a part of an online social support community, and at times can take up the mantle to provide crucial information~\cite{joglekar2018online}. My work with the suicide watch support community also shows that there are quantifiable structures of support which exhibit a patient centric structure of engagement. I foresee that with the help of A.I. models trained on these empathic interactions, we could one day see virtual support groups which are supported by the healthcare provider and allow self and group management of chronic conditions. Overall, I see that empathic signatures in communications, can be learnt at scale, if we can pinpoint where exactly such interactions are happening. And I think my work paves a path forward for that to happen. 


\section{Perceptive urbanism}
Urbanism is a term used commonly to describe works that deal with problems, phenomenon, mechanics or evolution stemming from urban life.
\begin{itemize}
    \item Cities are dynamic
    \item sturctures in cities are driven by human choices, and these choices are rooted in social and physchological aspects of how one percieves a city.
    \item through my work, we saw that a common idea of beauty can be leveraged to do small nudges to the overall look and feel of urban spaces. 
    \item but these perceptive triggers are hypothesized to affect several other aspects of human life. 
    \item green spaces and depression
    \item access to services and powerty
    \item drug and disease prevalence and its links to urban design 
    \item effects of climate change on how cities grow and thrive in an ever changing environment. 
\end{itemize}
All these questions, when looked at through the lens of crowd perceptions, could result in much explainable and actionable answers. I aim to use cutting edge data modelling techniques to contribute to the knowledge in this domain 



\section{Conclusion}

In the hindsight this work has been a result of series of fortunate accidents, which allowed me to explore and exploit interesting topics in the fields of information retreival, network science and at times social science. Being a computer scientist in today's age, I believe is as much an exercise in a broad understanding of today's social problems, as is it about technical details and methods to solve them. My Ph.D. has given me an unfettered opportunity to spend my time and resources in broadening my horizons. I hope I keep the pursuit alive in the due course of my career. 