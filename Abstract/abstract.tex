% ************************** Thesis Abstract *****************************
% Use `abstract' as an option in the document class to print only the titlepage and the abstract.
\begin{abstract}
The idea of ``wisdom of the crowds'' has been tested time and again when it comes to applications like recommendation while shopping items or visiting places or listening music. The underlying principle here is that \textit{subjective preferences tend to follow deterministic patterns, when looked at in large sample sizes}.
In this dissertation, I examine the thesis "Can we quantify entities of subjective nature, if the data is large enough, and originates from human responses?"

With the ever pervasive nature of the internet, we as a society have started treating the online spaces not only as a tool to access information seamlessly, but also as a natural extension of the self. We spend more time than before as a part of a larger networked community, exchanging thoughts, debating ideas, expressing creativity and ``socializing''. We also sometimes indulge in expression of human emotions like empathy, anger, sadness and sometimes seeking help. 
%The presence of these extended set of interactions with the web intuitively provokes the question: "Can we build pipelines that look at data from these interactions, to quantify, learn and intervene using subjective entities ?"
In this dissertation I develop data driven pipelines with the aim to quantify perceptions of subjective qualities, through two case studies. 
In the first study I analyse on-line formal networks where interactions between humans are purely with the aim of helping each other. I develop frameworks to abstract out the structure of these interactions. I then delve into investigating presence of perceived support by finding discriminative local structures in these abstractions. I reason about these local structures using established cross disciplinary theories. This informs my analysis about the nature of peer to peer support in these communities and paves the way to do actionable intervention in the area of peer support in online networks. I then develop metrics to quantify local and global signatures of supportive conversations, thereby establishing a framework to measure the subjective experience of support on the web.

In the second study, I investigate utility of perception of aesthetics of physical spaces, by developing a pipeline that capitalizes on the crowd sourced responses about urban aesthetics. I develop intervenable insights around urban beautification . I propose a deep-learning driven framework, which is able to quantify the perception of intangible abstract qualities like `beauty' through a crowd sourced rating of google street view images. 
I show that a general pattern of beauty in urban spaces can be learnt through a crowd sourced opinion and deep learning models. I further develop a generative model to simulate beautification of urban spaces. 
Through a detailed literature review of the field of urban design, I develop a measurement framework which can provide insights into the predictors of urban beauty on a case by case basis using well known urban design metrics. I validate the value of these metrics through expert survey and validate the interventions through crowd sourced perception experiment. 

As an over arching goal, through this dissertation, I try to build a repeatable framework for data driven pipelines that can capitalize on data that contains human perception signals. I propose that building custom representations that bind computational metrics with inter disciplinary ontologies is a necessity to build useful pipelines. 

$$ \lceil ~ ~  \rceil $$ 

\end{abstract}