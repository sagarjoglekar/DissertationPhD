% ************************** Thesis Abstract *****************************
% Use `abstract' as an option in the document class to print only the titlepage and the abstract.
\begin{abstract}
The idea of wisdom of the crowds has been tested time and again when it comes to applications like recommendations of items or places, which inherently taps into the likes and dislikes of the crowds. Interesting convergence has been shown to emerge when insights are driven by usage patterns of large number of people.

  
With the ever pervasive nature of the internet, we as a society have started treating the online spaces not only as a tool to access information seamlessly with atomic transactions, but also as a natural extension of the self. We spend more time than before as a part of a larger networked community, exchanging thoughts, debating ideas, expressing creativity, empathy and sometimes seeking help. The presence of these extended set of interactions over the web, implies that an intangible yet essential affective component is now driving our behaviour when we empathise with people online, or interact with creative or aesthetically pleasing content. At such a juncture, I ask the question "Can we build pipelines that look at these interaction patterns, to understand and utilize the perception driven behaviours of humans on the web?"

In this dissertation I explore the idea of "Perceptions of the crowds" through two case studies. I build systematic pipelines that quantify human perceptions in both cases interpreting interdisciplinary ontologies. In the first study I analyse on-line formal networks where interactions between humans are purely with the aim of helping each other. I develop frameworks to abstract out the structure of these conversations in computable structures. I then delve into investigating presence of supportive phenomenon by finding discriminative local structures in these abstractions. I reason about these local structures using established cross disciplinary theories. This informs my analysis about the nature of peer to peer support in these communities and paves the way to do actionable intervention in the area of peer support in online networks.

In the second study, I investigate utility of perceptions of physical spaces, by developing a pipeline that capitalizes on perception of urban aesthetics at the crowd scale, to develop intervenable insights around urban beautification. I propose a deep-learning driven framework, which is able to quantify the perception of intangible abstract qualities like `beauty' through a crowd sourced rating of google street view images. 
I show that a general pattern of beauty in urban spaces can be learnt through a crowd sourced opinion and deep learning models. I further develop a generative model to simulate beautification of urban spaces. 
Through a detailed literature review of the field of urban design, I develop a measurement framework which can provide insights into the predictors of urban beauty on a case by case basis using well known urban design metrics. I validate the value of these metrics through expert survey and validate the interventions through crowd sourced perception experiment. 

As an over arching goal, through this dissertation, I try to build a repeatable framework for data driven pipelines that can capitalize on data that contains human perception signals. I propose that building custom representations that bind computational metrics with inter disciplinary ontologies is a necessity to build useful pipelines. 

$$ \lceil ~ ~  \rceil $$ 

\end{abstract}