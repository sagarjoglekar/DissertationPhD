% ************************** Thesis Abstract *****************************
% Use `abstract' as an option in the document class to print only the titlepage and the abstract.
\begin{abstract}
Research in the field of recommendation systems have shown that
\textit{subjective preferences tend to follow deterministic patterns, when looked at data in large sample sizes}. This principle underpins several of our present day recommendation applications like online e-commerce or restaurant recommendations, or music recommendation. With the ever pervasive nature of the internet, we as a society have gone beyond treating the online spaces as a tool to access information, and have started treating them as a natural extension of the self. We spend more time than before as a part of a larger networked community, exchanging thoughts, debating ideas, expressing creativity and ``socializing''. We also sometimes indulge in expression of human emotions like empathy, anger, sadness and sometimes seeking help. We also at times behave like crowds, participating in entertainment and engagement channels, offering a piece of our attention budgets. At such a juncture, I examine the thesis \textbf{Can we quantify entities of subjective nature, if the data is large enough, and originates from human communities or crowds?}. In this dissertation I develop data driven methods with the aim to quantify subjective qualities, through two case studies.I investigate the utility of said methods in designing interventions to improve the online and offline spaces.

In the first study I analyse on-line spaces involving networked communities where interactions between humans are purely with the intent of helping each other. I develop frameworks to abstract out the graphical structure of these interactions. I then investigate presence of perceived support by finding discriminative local and global structures in these communities. Using established inter-disciplinary theories, I argue that these structures, which we call motifs, are the signatures of a supportive processes in online conversations. This informs my analysis about the nature of peer support in these communities and paves the way to do actionable interventions in the area of perceived support in online networks. 

In the second study, I investigate utility of crowd's perception of aesthetics in physical spaces. As such this utility indicates potential in designing better offline spaces using crowd perception. 
I do so, by developing a pipeline that capitalizes on crowd sourced responses about perception of urban aesthetics. I develop a deep-learning driven framework, which is able to quantify the perception of intangible qualities like `beauty of a space' through a crowd sourced rating of google street view images. I show that a general pattern of beauty in urban spaces can be learnt through crowd sourced opinion and deep learning models. I further develop a generative model to simulate beautification of urban spaces. Through a detailed literature review of the field of urban design, I develop a measurement framework which can provide insights into the predictors of urban beauty. I then develop the necessary tools to evaluate these metrics using computer vision techniques. I validate the value of these metrics through expert survey and also validate the interventions using crowd sourced perception experiments. 

Above all, in this dissertation, I contribute original frameworks and implementations of different approaches towards quantifying subjective signals, in communities and crowds. These approaches verify and validate several metrics developed for understanding subjective properties like perceived support and perceived aesthetics, at scale, which provides a path forwards for A.I. driven design of online and offline spaces.

$$ \lceil ~ ~  \rceil $$ 

\end{abstract}