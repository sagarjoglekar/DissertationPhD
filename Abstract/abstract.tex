% ************************** Thesis Abstract *****************************
% Use `abstract' as an option in the document class to print only the titlepage and the abstract.
\begin{abstract}
Research in the field of recommendation systems have shown that
\textit{subjective preferences tend to follow deterministic patterns, when looked at data in large sample sizes}. This principle underpins several of our present day applications like recommendations while shopping for items online or visiting places to eat or listening to music. With the ever pervasive nature of the internet, we as a society have gone beyond treating the online spaces as a tool to access information, and have started treating them as a natural extension of the self. We spend more time than before as a part of a larger networked community, exchanging thoughts, debating ideas, expressing creativity and ``socializing''. We also sometimes indulge in expression of human emotions like empathy, anger, sadness and sometimes seeking help. At such a juncture, I examine the thesis \textbf{Can we quantify entities of subjective nature, if the data is large enough, and originates from human responses?}. In this dissertation I develop data driven pipelines with the aim to quantify perceptions of subjective qualities, through two case studies. In the process, I provide a broad overview of how intangible subjective properties exhibit in data and develop metrics to quantify them. I also reason about the utility of said properties when it comes to developing interventions.

In the first study I analyse on-line formal networks where interactions between humans are purely with the intent of helping each other. I develop frameworks to abstract out the structure of these interactions. I then delve into investigating presence of perceived support by finding discriminative local structures in these abstractions. I reason about these local structures using established cross disciplinary theories. This informs my analysis about the nature of peer to peer support in these communities and paves the way to do actionable interventions in the area of peer support in online networks. 

In the second study, I investigate utility of perception of aesthetics in physical spaces, by developing a pipeline that capitalizes on crowd sourced responses about urban aesthetics. I propose a deep-learning driven framework, which is able to quantify the perception of intangible abstract qualities like `beauty' through a crowd sourced rating of google street view images. I show that a general pattern of beauty in urban spaces can be learnt through a crowd sourced opinion and deep learning models. I further develop a generative model to simulate beautification of urban spaces. Through a detailed literature review of the field of urban design, I develop a measurement framework which can provide insights into the predictors of urban beauty on a case by case basis. I then develop the necessary tools to evaluate these metrics using computer vision techniques. I validate the value of these metrics through expert survey and also validate the interventions using crowd sourced perception experiments. 

In the course of the work, I contribute original design and implementations of different data driven pipelines, which can be used to quantify signatures of subjective properties in a way that can drive interventions and impact real lives. 

$$ \lceil ~ ~  \rceil $$ 

\end{abstract}