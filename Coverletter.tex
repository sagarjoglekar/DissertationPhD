\documentclass[11pt,a4paper,roman]{moderncv}

\moderncvstyle{classic}
\title{From Communities to Crowds: Quantifying the subjective}      
\name{}{}



\begin{document}
\maketitle

Dear Prof. Joemon and Prof. Yulan, \\


I sincerely thank you for the time that you dedicated to examine my Ph.D., as well as for the insightful feedback. Over the past three months, I have tried to address it as much as possible. 
I hope this makes the document much stronger, and satisfies your expectations.


I would like to enumerate the changes that I have done to the dissertation document, in order to set the expectations and allow ease of readability. 

\section{Broad summary of edits}
The key feedback received pointed to the fact that I needed a better contextualization of the work, along with a finer reflection on the research questions and the findings. Across the dissertation, I have re-wrote sections in order to address the following points 
\begin{itemize}
    \item I have added an entire Background chapter (chapter 2) to the dissertation which walks through the different approaches to the problem of ``Quantifying the subjective''. The Chapter also lays out the fundamental work done in the two regimes written about in this dissertation viz. Communities and Crowds. 
    \item I have reformulated the research questions to be more specific. I have also added a fair bit of reflection at the end of each chapter, which reflects upon the key findings in each chapter and how do those help in answering the research questions. 
    \item I have fixed some problems with typos, missing text, and interpretability of figure captions across the entire document. 
    \item Whenever a method is proposed, or used , I have added text to describe the method as well as the reason behind the choice. 
\end{itemize} 


\section{Chapter 1}
\begin{itemize}
    \item Sharpened the description of DIKW pyramid. In order to minimize adding concepts without proper explaination, I have steered away from some of the problematic vocabulary
    \item I have clarified the reasons behind the labels of meso and macro signatures.
\end{itemize}

\section{Chapter 2}
This is an entirely new chapter with discussion about the previous works in the fields of quantifying the subjective. I have tried to tie together some background from the support theory and urban analytics, with the use case of computational quantification of certain subjective properties. 

\section{Chapter 3}
\begin{itemize}
    \item Refined the research question and discussed the implications of the findings at the end. 
    \item Fixed typos, missing text, and missing data in the chapter.
    \item All the metrics are introduced in the text, with references. These metrics are also discussed in the background chapter. 
    \item Made the text consistent with respect to the vocabulary of superusers.
    \item Improved data section with samples and processing information. 
\end{itemize}


\section{Chapter 4}

\begin{itemize}
    \item Improved the data section with specifics of collection methods and description of the data schema with example. 
    \item Discuss the significance of Frontpage posts as baseline.
    \item Corrected figure presentation. 
    \item Explained triadic motifs clearly and discussed the algorithm to compute the census
\end{itemize}

\section{Chapter 5}

\begin{itemize}
    \item Corrected research questions 
    \item Added details about data annotation for the pairwise comparison
    \item Added further details about the convolutional neural network model.
    \item Motivated the use of TrueSkill for converting pairwise comparisons into ordinal rankings.
\end{itemize}

\section{Chapter 6}

\begin{itemize}
    \item Reformulated research questions
    \item Added details about the GAN models along with the link to the code and the models for reproducibility. 
    \item Described the PlacesNet and Segnet models. 
    \item Fixed references to FaceLift. 
\end{itemize}

\section{Chapter 7}
\begin{itemize}
    \item Restated RQs and reflected on the results of all the works. Summarised the findings in relation to the RQs.
\end{itemize}

The changes done in the document are based on the feedback and my own judgement about the 

Best Regards,

Sagar Joglekar

\end{document}