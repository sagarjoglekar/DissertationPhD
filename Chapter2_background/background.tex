\chapter{Background}  %Title of the First Chapter 
\label{chap:background}
%
%\begin{quote}
%\textsl{If I have seen further it is by standing on the shoulders of Giants.} - Isaac Newton
%\end{quote}

Studying signals in user interaction data, where the interactions are  driven by affective triggers, has been an active topic of research~\cite{picard2003affective,pantic2007human,cambria2012sentic,adibuzzaman2013situ}. 
As such it is worth discussing the different aspects in which the community of researchers have explored this area. The key aspects in which I would like to place my work is in terms of quantification of the subjective signals from data that originates from web scale applications. 
The two case studies in this dissertation attempt to quantify two distinct subjective properties, viz 1) Social support and 2) Aesthetic perception.
Here we would try to first, explore the definitions of the concepts of social support and Aesthetic perception, and then examine the literature for methods, models and metrics.

\section{Part 1: From communities}
There has been a surge in the number of online communities, since the rapid adoption of social networks across the internet. The spectrum of types and applications of these communities is as abundant as the possible subjects discussed on the internet. 
These communities have become a dedicated space for online users to discuss about topical items. Often these communities take the shape of a forum, where topical threads are started by an original poster (OP) and a discussion commences on this post. The discussions could be in the form of a debate\footnote{\url{www.kialo.com}}, a banter\footnote{\url{www.reddit.com}} or a peer to peer topical discussion\footnote{\url{https://healthunlocked.com/blf}}. In this dissertation, we aim to understand the nature of social support and the signatures of social support which can be quantified from these online spaces. To that end we aim to look at forums which have been self certified to be dedicated for hosting supportive discussions.

\subsection{Online social support}
According to Shumaker et.al~\cite{shumaker1984toward} social support is defined as "an exchange of resources between two individuals perceived by the provider or the recipient to be intended to enhance the well‐being of the recipient."
A lot of work has been done in understanding how social support functions and influences people in distress in the offline world. For example, a meta review~\cite{dimatteo2004social} showed that adherence to a medical treatment is 1.74 times higher, if the patient hails from a cohesive family structure. Social support has also shown to be a crucial factor in the positive prognosis of patients suffering from chronic and long term conditions~\cite{sacco2006diabetes,peirce2000longitudinal,brown1986social,collins1993social,dunkel1984social,baron1990social}. All this literature evaluates social support from a psychological stand point, in that, it looks at how a patient/subject is perceiving support from its real world network (family, friends, doctors etc.) More over most of this work uses the qualitative frameworks and tools as a way to measure off-line social support. 

Social support, or the perception of help received from others, is a widely studied as a psychological resource used to cope with distress. 
Social support is generally classified into one of the following categories viz. Informational, Tangible , Network or Emotional~\cite{cutrona1992controllability}. These categories measure the nature of social support along the idea of exchange of resources between the person in distress and the one providing support. For example, an informational support could involve pure exchange of valuable information about dealing with an issue. Whereas as network support could purely allow the recipient to acquire a wider network of people through the support giver or a support platform (think societies like Alcoholic Anonymous). 

The online world is very good at filling the gaps where offline support groups may fail. That is, the online groups, if designed in the right way, could provide essential informational, network, and at times emotional support at the click of a button. The internet makes both information and user networks, easily accessible. The utility of such spaces can be believed because of the evidence such as: online support groups being associated with quality of life~\cite{idriss2009role,nambisan2011information,coulson2005receiving}, improved control over additions~\cite{wood2009evaluation}, improved triaging with suicide ideation~\cite{languageChoudhury} and mental health issues~\cite{kummervold2002social}. 

Despite the interesting evidence, the common theme in most of these works is that they all take either a qualitative approach to understanding online support, or a language driven approach~\cite{de2014medical,de2016stroke,de2017adolescents}.
Most methodologies either utilize expert knowledge to dissect what is being said on these forums, or a natural language approach to understand the key language patterns on these forums. Either ways, the key missing bit in this picture is understanding how and where do the individual users fit in. How do they help in keeping the entire support community functioning. More over, we do not have any understanding about the structure of a supportive conversation online from the perspective of a user in distress. This is especially interesting, since there are obvious offline markers of a supportive conversation, whether it being a group, or a peer to peer setting. Not just that, but there are several therapy strategies like Communication accommodation~\cite{coupland1988introduction} or group psychotherapy~\cite{yalom_theory_1995}. So it is worth investigating these aspects of online support communities.

For this exact reason, \textbf{RQ1, RQ2 and RQ3} would make progress towards understanding the dynamics, local and global structures of online support communities.
 
\subsection{Social networks, support, and metrics}
Modelling and studying online social networks, through the lens of complex network theory and network physics has been an active area of research since the 1980s. The idea of looking at (offline) social structures as social networks was quite prominent in the fields of sociology~\cite{scott1988social}, but the most important leap in this field came with the advent of online social networks. This opened up new doors in measuring and understanding how humans form networks, and do so at scale through the medium of online networks~\cite{mislove2007measurement}. The first of these works~\cite{mislove2007measurement} explored the idea of looking at large scale user graphs to measure and evaluate a lot of big picture attributes about online social networks. In that, they measure the long tailed nature of social links, the symmetric nature of social links and the overall sizes of connected components.
A connected component in a social graph $G(V,E)$ ) -- where $V$ are the vertices or users and $E$ are the edges between them -- is a set of nodes $\{C\}$ where all nodes $\{c_i \in C\}$ have connected paths to all other nodes in $\{C\}$. The size of the largest connected component was often used as a proxy for the connectedness of a social graph~\cite{myers2014information,traud2012social,woodhouse1994mapping}.
Other global scale metrics explored for understanding social networks were degree distributions~\cite{muchnik2013origins,newman2002random, kossinets2006empirical}, clustering coefficient ~\cite{opsahl2013triadic, toivonen2006model} and centrality~\cite{opsahl2010node, borgatti2009network}. All these metrics aim to look at how nodes(users) in a social graph group together or how do they interact with each other as the size of the network increases. 
These insights help network physicists model how information or a contagion diffuses in a connected community. However , despite the large number of interesting works that look at network structure, there have been limited progress in using these tools to understand the nature of social support in online networks.

In the offline world, there have been some studies that look at the ego network of a person of interest, to infer the nature of social ties they have. In that, they look at the transitive nature of social networks around a person~\cite{golden2009social,doi:10.1177/104649647100200201, lin2001social}. This means, how many friends of a person, are friends among themselves. A completion of this triad -- which implies that friends of my friends are my friends -- is called a triadic closure.
In the online world, the theory of social capital and triadic closures were operationalized in terms of triadic census~\cite{faust20077,faust2008triadic}. The census simply profiles any given network by counting individual instances of the individual sub-graphs made of two, three, or in some cases four nodes. These sub-graphs are called motifs.
The importance of triadic motifs in social network research has been stressed so much so that in the words of  Holland and Leinhardt~\cite{holland1977method} 
\begin{quote}
    ``The essential issue of any notion of structure is how the components are combined, not the components themselves...this issue amounts to the proposition that the lowest interesting level of structure...is the level of triples of nodes—the triadic level''
\end{quote}


From the studies discussed here, the link between human to human interactions and their emergent network structure are quite evident. For this reason it is natural to extend this link to explore how perceived social support manifests in these interactions. In this dissertation, I aim to first quantify how network structures in support communities evolve, and second, discover the signature structural properties of these interactions over support communities.

\section{Part 2: From crowds}
The second part of the dissertation explores new methods to use crowd's opinions, in order to build models of our subjective perception of urban spaces. The idea of using crowd based annotations or opinion mining has been popular and has been exploited in the recommendation systems literature for a while. Here I would introduce some background about the idea of crowd sourcing, how it applies to quantifying the subjective, and how it has been used to understand urban spaces and cities. 

\subsection{Crowd sourcing and the subjective}
Crowdsourcing has been an important part of the computer science research for the past decade. The idea of crowdsourcing was first brought to attention by Jeff Howe~\cite{howe2006rise}, where he introduced the idea of a logical equivalent of outsourcing -- which is sending the jobs outside an area, where labour prices are competitive-- but for more transactional and atomic tasks. This idea was quickly adopted by the academic community, right after the inception of services like Amazon Mechanical Turk or Crowdflower~\cite{paolacci2010running}.

The natural extension of this new method was to use it for annotating large quantities of data. These annotations generally dealt with labels of objective nature, such as objects~\cite{vondrick2013efficiently}, relationships between objects~\cite{krishna2017visual}, or annotating textual data like named entities~\cite{finin2010annotating}.

The key idea behind crowdsourcing is to get a cognitive input about unlabelled data by incorporating opinions of hundreds of ``crowd-workers'' exchange for money. This is done in order to annotate the data with the most accurate objective labels, that come from a human annotator. This data then becomes the training set for a downstream machine learning algorithm . These algorithms would then learn the task of classifying unseen data into their respective correct labels. 

But the very fact that a cognitive process is at the root of the annotations implies that this framework can even be applied to subjective properties, provided that the annotators can arrive at a consensus, and we have large enough samples. The most appropriate use case is that of annotating expressions or affects of humans in videos or images. Despite the subjective nature of perceived affect, most neuro-typical humans tend to agree on what constitutes the expression of anger, sadness, happiness , disgust etc. In that spirit, several studies~\cite{tavares2016crowdsourcing,katsimerou2016crowdsourcing,kim2016vinereactor} tried to build machine learning models that could detect emotions from facial expressions, using data annotated by the crowd.

Apart from building a model of human affects from faces, crowdsourcing was also applied to the area of quantifying the actual affective stimuli in content. That is it tries to quantify the intangible property of a content that stimulates evocation of a particular emotion in the consumer of that content. For example, it is worth asking ``Which emotion does an image of a sunset evoke in the human seeing it?''. This question goes one level deeper by trying to quantify that which evokes positive or negative sentiments. To that end Sentibank~\cite{SentiBank} explored this idea by training a deep learning model on Flicker images which were annotated for evoking positive or negative sentiments. The same team extended it to analyse how the evoked emotion changes as a function of culture and language~\cite{pappas2016multilingual}. Indeed they found that these evoked emotions are also dependent on the language, culture, and other social properties of the annotator. Although these works exposed some limitations in the approach of quantifying the subjective, they also showed that by and large, these techniques work if the data is large enough and there is considerable consensus among the annotators on the topic of the annotated subjective property of the data.


\subsection{Crowds and the cities}
So far, the most detailed studies for quantifying the perceptions of urban environments and their visual appearance have relied on personal interviews and the observation of city streets: for example, some researchers relied on annotations of video recordings by experts~\cite{sampson04seeing}, while others have used participant ratings of simulated (rather than existing) street scenes~\cite{lindal2012}. 
But since the advent of services like the Google street view and Open Street Map, the Web has now been used to survey a large number of individuals. Place Pulse is a website that asks a series of binary perception questions (such as `Which place looks safer [between the two]?') across a large number of geo-tagged images~\cite{salesses2013collaborative}. In a similar way, Quercia \emph{et al.} collected pairwise judgments about the extent to which urban scenes are considered quiet, beautiful and happy~\cite{quercia2014aesthetic} to then recommend pleasant paths in the city~\cite{quercia2014shortest}. Another study~\cite{seresinhe2015quantifying} presented the annotators with a 10 point scale, which they would use to score a place(Street view) for its aesthetic beauty. 
All these studies relied on the crowds, in that the annotators were completely disconnected from each other, and their ratings were purely based on their exposure to the image or an urban scene. An important caveat here, as in case of multi lingual sentibank~\cite{pappas2016multilingual}, is that the cultural and social background of the annotator would play a role in how they perceive an urban scene.
But on average, these annotations proved very useful in understanding something as subjective as the perception of safety, beauty, or memorability in urban spaces. 

This can be indicated by the fact that lately deep learning techniques have been used to accurately predict urban beauty~\cite{dubey2016deep,seresinhe2017using}, urban change~\cite{naik2017computer}, and even crime~\cite{DeNadai16,arietta2014city}.  Recent works have also showed the utility of deep learning techniques in predicting house prices from urban frontages~\cite{frontage}, and from a combination of satellite data and street view images~\cite{law2019take}.

All the studies discussed above were successful in quantifying the subjective properties of an urban scene using predictive machine learning models. But there is a significant gap between predicting and explaining the prediction in order to guide interventions. This explainability problem is prevalent in almost all applied machine learning systems. In this dissertation, I attempt to make some progress on the front of explaining the reasons behind perceiving an urban scene beautiful or ugly. These explanations also come in the form of urban design metrics, which can guide interventions from the practitioners of this field.

\section{Discussion}

The title of my dissertation enumerates the two regimes --communities and crowds-- under which I explore the problem of capturing the signatures of perceived subjective properties, using customized metrics and models. The over arching thesis has always been understanding how human perceptions guide our actions on the web scale, and how these actions leave behind traces of the subjective triggers.

In the following chapters, I will discuss the different methods, metrics, and models that are developed in order to make progress in answering the central thesis of this dissertation.


