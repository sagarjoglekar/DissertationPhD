%*******************************************************************************
%****************************** Second Chapter *********************************
%*******************************************************************************

\chapter{Online health social networks}

Online communities have the potential to influence health and health care. Recent studies have suggested that the participation of people with long-term conditions (LTCs) in online communities (1) improves illness self-management [1], (2) produces positive health-related outcomes [2-4], (3) facilitates shared decision-making with health care professionals [5,6], and (4) may even reduce mortality [7].

There is also evidence that self-management support interventions can reduce health service utilization [8,9].

Online communities have experienced an upsurge in popularity among people with chronic respiratory conditions such as cystic fibrosis [10], asthma [11], pulmonary hypertension [12] and chronic obstructive pulmonary disease (COPD) [13]. More than 15 million people in England suffer from a long-term condition or disability, and they account for at least 50 percent of all general practitioner appointments [14,15]. Thus, assessing how these online communities function and evolve can have important implications for health care provision.

This form of “user-led self-management” of LTCs bears similarities with the “expert patient” model, an approach to self-management of LTCs produced by the United Kingdom (UK) Department of Health in 2001 [16]. Evidence of the effectiveness of conventional off-line self-management programs based on the expert patient model, though, has been weak [17]. Clinic-based self-management programs often failed because of: (1) lack of awareness and engagement among patients and staff, (2) failure to consider low health literacy or cultural norms, (3) lack of attention to the need for family and social support, and (4) a fragmented approach to the provision of health and social care [18]. Although online health communities can be seen as an extension of the expert patient model, network effects, in addition to the online disinhibition effect [19], make them a distinct and unique complex intervention mechanism.

On average, one in four people with an LTC who use the Internet tries to engage online with others with similar health-related concerns [20]. In particular, it has been suggested that the value of participating in an online community lies in the possibility of gaining access to a range of people and resources quickly, easily [21], and anonymously [4], as well as obtaining tailored information and emotional support [1,22-26]. However, most of this evidence comes from qualitative studies [1,27], whereas only recent years have witnessed an increasing interest in quantitative assessments of online communities as intervention mechanisms [28-33]. Recent studies have been concerned with the users’ unequal contributions and engagement patterns, and with the role of superusers. However, the contribution of superusers to the sustainability of online health communities and their structural properties remains mostly unclear.

The potential future integration of online health support systems with formal health care provision should be underpinned by a better understanding of how they are used and by evidence of their effectiveness. Indeed, as suggested by the Medical Research Council [34], integrating online support systems with the more traditional health care provision would require the identification and comparative assessment of potential alternative intervention mechanisms.

An expanding body of literature concerned with social network analysis has examined the structural patterns of relations among interacting actors and the social mechanisms that enable them to gain access to valuable resources [35]. There is also increasing evidence that network approaches can be applied to understanding the users’ “expertise” [36], their interactions, and network effects on health-related outcomes in online health communities [37,38]. Uncovering the mechanisms underlying the formation of successful social networks requires a study of how online connections among people, namely the social ties or links, emerge and evolve, and how groups of individuals gradually grow in membership and become interconnected with one another. These processes of tie creation and group formation in online patients’ communities are still mostly unexplored [1].

In this study, we performed a network analysis of the structure and dynamics of two online communities of people with LTCs. We chose the Asthma UK and the British Lung Foundation (BLF) communities as an exemplar of such communities because their users typically suffer from chronic respiratory conditions. In particular, while Asthma UK users typically suffer from a respiratory condition characterized by variable and recurring symptoms, BLF users represent a more heterogeneous population of participants affected by different diseases linked to chronic symptoms of breathlessness (eg, COPD, pulmonary fibrosis, cystic fibrosis, and lung cancer).
We aimed to uncover and understand how these communities function and evolve, and the role that some users have in maintaining integration and cohesion (see Textbox 1 for research questions). Ultimately, this study provides evidence for gauging the effectiveness of different interaction patterns and the users’ structural positions and their potential for enhancing and sustaining health online communities as scalable self-management support interventions.\textbf{}




\section{Dataset and properties}
Data were collected by HealthUnlocked [39], the online platform provider of the Asthma UK and BLF communities. Registered users can choose to either write posts publicly or send private posts to one another. In the latter case, posts are shared between 2 users only, whereas when posts are written publicly, a large number of users can become connected through threads of posts. Only posts that were shared publicly were collected and analyzed. For this study, user identifiers (IDs) were anonymized by HealthUnlocked, and no demographic information was collected. The data sets included posts and their metadata (ie, the anonymized user ID numbers), user roles (eg, user, administrator, or moderator), date of posting, the hierarchical level of the post within the corresponding thread, and the dates in which the users joined and left the community. Both communities were moderated, and HealthUnlocked moderators (identified through metadata linked to posts) were included in the analysis to assess their contribution and compare it with other users. Online communities on the HealthUnlocked platform benefit from additional functionalities compared to other online forums, such as built-in patient groups that moderate the content. In particular, the content accessed by users is tailored to their interests, and profiles highlight users’ condition, chosen community, medications and treatments they use or find interesting. No data were collected on participants’ characteristics, though only people declaring themselves to be older than 16 years were permitted to create an account and take part in the online communities.


\section{Graphs: A primer}



\section{Activity patterns of users}

We looked at the number of users, the number of posts and connections per user and posting frequency. A connection (ie, a tie, link, or edge) was established from one user to another when the former replied to a post by the latter (see Textbox 2 for network analysis terminology). The pattern of connections generated over time through the cumulative number of posts and replies was examined. We were interested not just in the number of posts and responses but in who responded to whom, and when. To this end, we used social network analysis [40] to visualize and study the structure of the relationships between users. Both visualization and analysis were conducted using the Gephi software. The network analysis was carried out through additional custom computer code in python. Descriptive analysis of the networks (ie, number of users, posts, and posting frequency) were calculated using the Pandas library, an open source library providing data structures and analysis tools for the Python programming language.

As a result of the small percentage of users who wrote posts to a disproportionally high number of users, the users’ activity showed long-tailed distributions. Therefore, our analysis was based not only on means and standard deviations but also on medians.

To uncover time patterns in posting activity, we used Fourier transforms of the time series of the users’ activity [46], a known method used for the analysis of signals. Through Fourier transforms, we identified the frequency components, called harmonics, that together made up the posting activity stream. In other words, we regarded the posting activity over the entire observation period in both communities as a complex signal and identified the frequency components that made up such a signal. This analysis was performed using custom code in Scipy, a Python-based scientific computing library.

The “rich-club” coefficient is a metric designed to measure the extent to which well-connected users tend to connect with one another to a higher degree than expected by chance [43]. To this end, for each value k of a node’s degree (ie, the number of other users a given user is connected with), we computed the ratio between the number of actual connections between nodes with degree k or larger and the total possible number of such connections [47]. We then divided this ratio by the one obtained on a corresponding random network with the same number of nodes and degree distribution (ie, the probability distribution of the degrees over the whole network) as the real network, but in which links were randomly reshuffled between nodes. Thus, the rich-club coefficients may take values lower or higher than 1, depending on whether the real network has a higher or lower tendency to coalesce into rich clubs than randomly expected. In particular, networks that display a high rich-club coefficient (ie, greater than 1) are also said to show a “rich-club effect,” namely the tendency to organise into a hierarchical structure in which highly connected nodes preferentially create tightly knit groups with one another, thus generating exclusive clubs of (topologically) rich nodes, as illustrated in previous work [48].

In our study, superusers were defined according to their cumulative activity over the entire observation period. In total, we identified 400 superusers. To uncover how many superusers were active within each week, we detected how many unique users, among the 400 identified over the entire period, were active within that time window.

Following Zhang et al [36], the “z-score” was used as a proxy for users’ expertise. According to this measure, replying to many questions suggests one’s expertise, while asking questions indicates lack of expertise. In our analysis, we treated anyone starting a thread as a help-seeker, and anyone commenting on the thread as a help-giver [36]. Accordingly, the proposed z-score aims to capture the combined help-seeking and help-giving patterns. To this end, for each user, we measured how many standard deviations the observed total number of the user’s help-giving posts lies above or below the expected number of help-giving posts for the whole system. We extended the approach proposed by Zhang et al by empirically assessing the probability of posting and answering a question across all users over the entire observation period. In the BLF community, we found that the probability of answering is Pa=2/3, while the probability of posting is Pq=1/3. We assumed a Bernoulli process of posting an answer or a question to the forum, with probabilities defined as above. The z-score for a given user i was calculated according to equation (a) in Figure 1, where ai refers to the total number of answers user i posted to the forum, qi is the total number of questions user i asked in the forum, and ni=ai +qi is the total number of messages posted by user i.
To obtain $Zscore_i$, let us define a random user that posts the same total number of messages nrandom to the forum as user i (ie, nrandom=ni). We would expect this random user to post an average number of answers to the forum given by equation (b). Plugging in the value of $Pa=2/3$, we obtained equation (c). Similarly, we would expect the random user to post answers with a standard deviation given by equation (d). Plugging in the value of $Pa=2/3$, we obtained equation (e). To measure how many standard deviations above or below the expected random value a user i lies, we then computed Zscorei according to equation (f). Plugging in the values of $\mu_{random}$ and $\sigma_{random}$, we obtained equation (g). Finally, by substituting $ni=ai +qi$, we obtained equation (h). 

\section{Dataset charactarization}
The data sets span, respectively, 10 years for the Asthma UK and 4 years for the BLF communities (see Table 1).

Despite the shorter time span, as a result of the larger number of users, the number of posts in the BLF community was higher than in Asthma UK, namely 875,151 compared to 32,780 respectively. Moreover, BLF users wrote a higher number of posts per user and were connected with a higher number of other users when compared with people in the Asthma UK forum (see Figure 2). In both communities, 60\%-70\% of registered users wrote no posts (ie, they were lurkers). Users who wrote more than one post contributed with a median of 8 (range 2-8947) and 5 (range 2-1068) posts in the BLF and Asthma UK communities, respectively.

The number of official moderators among the highly active users was negligible; there were no moderators in the top 5\% contributors to BLF and only 2 in the top 5\% for Asthma UK. Thus, our network analysis predominantly reflects content originated from registered users.

When classified according to posting activity (ie, number of posts written to the forum), the top 5\% users contributed to a substantial proportion of all posts: 58\% and 79\% in the Asthma UK and BLF communities, respectively. Superusers were those who made a high number of connections with other users in both Asthma UK and BLF communities (see nodes of large size in Figure 2). Asthma UK superusers made a lower number of connections than BLF ones. The posting activity of these superusers will be analyzed in more detail in subsequent sections.
\subsection{Activity}
Posting Activity

The cumulative number of messages posted grew uniformly over time in the BLF community. By contrast, in 2015, the Asthma UK forum witnessed a substantial increase in posting activity, at a time coinciding with its move to the HealthUnlocked platform (see Figure 3A and B). This increase in activity can be attributed to the online community functionalities offered by HealthUnlocked, as described in the Methods.

The number of posts per user per week oscillated around a decreasing and an increasing trend (Figure 2C and D), while at the same time the number of posts always went up over the study period (Figure 1A and B). This suggests that there were intervals of time during which the rate of increase in new users was larger than the rate of increase in total posts. Moreover, in the Asthma UK forum users wrote according to two time patterns—they posted at an interval of 1-20 days or 6 months (Figure 4A), while those in the BLF community at an interval of 2 days (Figure 4B).

As more users joined the communities and connected to one another through online posts, distinct groups of connected users started to emerge. These groups, called network components (see Textbox 2), have fundamental implications for the effectiveness of processes of network dynamics such as information diffusion [49]. In a relatively short period, both communities underwent the formation of the “largest component” of connected users, namely a connected subset of users whose size increasingly outgrew the size of all other components (see Figures 1 and 4, and Multimedia Appendices 1 and 2). The largest connected components in both communities included 60%-100% of users.

Figure 5 suggests that, as time went by, the number of forum participants and their posting activity increased, and the proportion of users who were part of the largest components decreased. This finding was expected because the number of posts also rose exponentially, yet at times at a lower rate than the one at which new users joined the communities (see Figure 1C and D). It, therefore, became more difficult for the network to self-organize into a connected component that would include 100\% of the users. Figure 5A also shows that around week 450, when the forum moved to the HealthUnlocked platform, a larger fraction of users began to join the largest connected component, thus highlighting the role that the new online platform played in strengthening the connectedness of the network (see also Figure 3A and B).
\section{Propensity to help} 


\section{Not like conventional networks: Anti-rich club effect}


\section{Key takeaways, possible interventions}

